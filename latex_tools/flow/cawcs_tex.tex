\documentstyle[11pt,twoside]{article}
\pagestyle{plain}
\title{Gebruik van \TeX\ en \LaTeX\ op het CAWCS}
\author{Leo en Marion van Geest}
\date{28 februari 1991}

%Indien de style opties A4 en A4wide niet aanwezig zijn dienen de
%volgende statements actief te worden 
%\topmargin 0 pt     %    Nominal distance from top of paper to top of page
%\textheight 46\baselineskip
%\advance\textheight by \topskip
%\oddsidemargin 0.125 in    %   Note that \oddsidemargin = \evensidemargin
%\evensidemargin 0.125 in
%\marginparwidth 0.75 in
%\textwidth 6.125 in % Width of text line.
%Einde A4 en A4wide opties

\hyphenation{do-cu-men-ta-tie om-ge-ving}

\begin{document}

\maketitle
\parindent 0pt
\parskip\parsep

\begin{center}
{\LARGE Wij zijn niet ge\"{\i}nteresseerd in \TeX\ en \LaTeX!}
\end{center}

\section{Inleiding}

Het CAWCS is het Centrum voor Automatisering van Wapen en Commando Systemen der
Koninklijke Marine. Een marinebedrijf dat de operationele software ontwikkelt
welke draait aan boord van boven- en onderwaterschepen en vliegtuigen. Daarnaast
worden voor dezelfde platformen trainers gemaakt waarop bemanningen aan de wal
onderricht krijgen.

Het CAWCS heeft ongeveer 150 medewerkers waarvan 100 zich daadwerkelijk met
software ontwikkeling bezighouden. De overigen behartigen de operationele kant
van de systemen of bieden administratieve en technische ondersteuning.

De auteurs zijn leden van de in totaal uit 5 personen bestaande sectie Support
van de afdeling Systeem Ontwikkeling. De taak van deze sectie is het instant
houden en onderhouden van de ontwikkelomgeving, ondersteuning van gebruikers,
het ontwikkelen van nieuwe en onderhoud aan bestaande tools, de evaluatie,
installatie en in gebruik name van software producten van derden etc.

\section{Ontwikkelomgeving}

De omgeving waarbinnen de software ontwikkeling plaats vindt bestaat uit een
cluster van 6 VAX computers waaraan 19 VAX werkstations gekoppeld zijn, allen
met een VMS operating systeem, alsmede enkele militaire computers. Gebruikers
hebben de beschikking over VT200 en VT320 terminals, een aantal LN03 laser
printers en enkele PostScript printers.

De voornaamste software pakketten die gebruikt worden bestaan uit Teamwork voor
analyse en design, Oracle data base management systeem, Ada als programmeertaal
voor de VAX computers, ondersteund door de complete toolset van DEC (VAXset) en
Mars/Mascot met RTL/2 voor de militaire computers. 

En, het zal U niet verbazen, \TeX\ en \LaTeX\ met een heleboel gerelateerde
tools.

\section{\TeX\ en \LaTeX\ binnen het CAWCS}

\TeX, \LaTeX\ en de gerelateerde tools worden binnen de afdeling Systeem
Ontwikkeling van het CAWCS gebruikt voor het genereren van allerhande
documentatie. De nadruk ligt hierbij op twee aspecten: \LaTeX\ en gebruik.

Software productie is waar het om draait en daarbij behoort een aanzienlijke
hoeveelheid documentatie. \LaTeX\ wordt gezien als een tool waarmee het
produceren van documentatie vereenvoudigd wordt terwijl de kwaliteit verbetert.
Het werkt productie verhogend. \TeX\ daarentegen wordt door velen als `te
moeilijk voor het dagelijks gebruik' ervaren. 

Binnen het CAWCS zijn slechts enkele \TeX\ kenners aanwezig. Samen met de
medewerkers van de sectie Support, zorgen zij voor uitbreiding van de \LaTeX\
omgeving. Alleen uitbreiding, want naast kwaliteit is continu\"{\i}teit \'e\'en
van de belangrijkste aspecten binnen een productie omgeving. Wat vandaag kan
moet morgen ook kunnen. Het mag wel sneller, mooier, uitgebreider maar niet
anders.

\section{Realisatie}

Alle \TeX\ gerelateerde software betrekt het CAWCS uit de DECUS \TeX\
collection. De sectie Support is verantwoordelijk voor kwaliteit en
continu\"{\i}teit en selecteert vooraf een subset uit de aangeboden styles,
tools, etc. afgestemd op de ontwikkelomgeving en de gebruikerswensen.

In verband met toekomstige upgrades, wordt zoveel mogelijk gekozen voor
standaard producten. Wijzigingen beperken zich tot het corrigeren van eventuele
fouten, indien mogelijk, kleine aanpassingen aan styles en inpassing in de
ontwikkelomgeving, gebruikers interface, opslag structuren etc. De meeste
aangeboden styles en opties worden zonder meer overgenomen.

Presentatie aan de gebruikers geschiedt in de vorm van manual pages en users
reference manuals, verzameld in een zogenaamd Support Manual. In dit manual
wordt alle algemene CAWCS software beschreven en de hele \LaTeX\ omgeving is
hierin ook ondergebracht. Het Support Manual wordt overigens ook
geproduceerd met \LaTeX. Hiervoor zijn speciale styles ontworpen, zowel voor de
manual pages als voor de reference manuals.

\section{De Selectie}

Op het CAWCS worden naast \LaTeX\ en \TeX\ de volgende tools in meer of mindere
mate gebruikt:

\nobreak
\begin{itemize}
\item \TeX\-achtigen
    \nobreak
    \begin{itemize}
    \item Bib\TeX\ 
    \item Sli\TeX\ 
    \item Glo\TeX\ 
    \item Idx\TeX\ 
    \end{itemize}
\goodbreak     
\item DVI verwerking
\nobreak
    \begin{itemize}
    \item DVI2LN3   (conversie naar LN03 code) 
    \item DVIALW    (conversie naar PostScript)
    \item CRUDETYPE (conversie naar text file) 
    \item DVI2TTY   (preview op text terminal) 
    \item XDVI      (preview op VAX station)   
    \item DVITOVDU  (preview op verschillende terminals)     
    \item DVITYPE   (DVI leesbaar maken)       
    \end{itemize}
\goodbreak
\item Algemeen
\nobreak
    \begin{itemize}
    \item SPELL (spelling checker met \TeX\ know-how)
    \item LSEDIT environment en section files (ter ondersteuning van het
    intikken t.b.v. \LaTeX, Bib\TeX\ en Sli\TeX\ en met de mogelijkheid \LaTeX\
    te starten vanuit de editor en fouten te `reviewen')
    \item RNOto\TeX\ (conversie Digital Standard Runoff naar \TeX)
    \item TR2\TeX\ (conversie Troff naar \TeX)
    \end{itemize}
\end{itemize}


\section{De Uitbreidingen}

De gekozen subset is op het CAWCS uitgebreid met styles en opties die aansluiten
bij de wensen die ontstaan als men zich met systeem ontwerpen en programmeren
bezighoudt. Bovendien dient men zich voor wat betreft de documentatie van de
operationele software te houden aan bepaalde standaards.

Dit laatste heeft geleid tot twee document styles, QADoc en QADocLand, die met
name de voorgeschreven layout ondersteunen. Beide styles zijn in principe
gelijk, met dien verstande dat QADoc portrait mode en QADocLand landscape mode
ondersteunt. De styles kenmerken zich door een grote bladvulling, een standaard
pagestyle met daarin een kader en een vaste heading, de mogelijkheid een footer
met daarin de document classificatie op nemen, paginanummering per hoofdstuk en
een nesting van subsections tot maximaal 7 levels. De heading bevat een titel,
subtitel, documentnummer, hoofdstuk- en pagina nummer, datum en auteur. Als
gevolg hiervan bleek het nodig de DVI converters voor LN03 en PostScript te
voorzien van een vaste set qualifiers/opties en iets gewijzigde pre-amble files.
Er is een omgeving gecre\"eerd waarbinnen \LaTeX\ draait en waarin de gebruiker
afhankelijk van de gebruikte documentstyle, de daarbij behorende DVI conversie
kan kiezen.

Vanuit de hoek van de programmeurs is de wens gekomen om in de module design
documentatie, verschillende data en controle structuren te kunnen opnemen.
Hieruit zijn een aantal style opties voortgekomen die het mogelijk maken flow
charts, Nassi-Schneidermann diagrammen, data structuren en bitpatronen in een
document te defini\"eren.

Voor data structuren en bitpatronen is een \LaTeX\ environment gemaakt
waarbinnen bits, bytes en woorden gespecificeerd kunnen worden, voorzien van de
nodige tekst, en waar een tekening uit komt van de gespecificeerde
datastructuur.

Voor flow charts en Nassi-Schneidermann diagrammen is gekozen voor specificatie
in de vorm van pseudo-code. Alle bekende structuren als if-then-else, while-do,
case, repeat-until en de acties zijn gedefinieerd als commando's. Deze dienen
gebruikt te worden binnen een specifiek commando waarin het totale diagram
gedefinieerd wordt. Overigens zijn beide styles geschreven in \TeX.

Ten behoeve van alle bovenstaande, veel gebruikte, styles en opties is de
Language Sensitive Editor omgeving uitgebreid met ondersteuning hiervoor. Deze
ondersteuning is grotendeels ondergebracht in aparte files, die zonder problemen
toegevoegd kunnen worden aan de uit de \TeX\ collection afkomstige
ondersteuning.

Een andere ontwikkelde style is Manual, ten behoeve van het al eerder genoemde
Support Manual. Hierin zijn commando's opgebracht die het mogelijk maken manual
pages en users reference manuals te produceren. Een style optie Listing maakt
het mogelijk source code op te nemen binnen deze style, wat vooral gebruikt
wordt om specificaties van standaard (Ada) packages te documenteren.

Als laatste, de enige pure \TeX\ applicatie, Cards, een tool om eenvoudige
visitekaartjes te maken.

\section{Integratie en andere toepassingen}

Behalve uitbreiding van de bestaande \LaTeX\ omgeving, is er ook gezocht naar
toepassingen waarbij \LaTeX\ gecombineerd wordt met andere software
ontwikkelings tools die bij het CAWCS in gebruik zijn.

E\'en van de meest productieve inspanningen op dat gebied, heeft geleid tot een
uitbreiding van de Document Production Interface (DPI) van Teamwork. Teamwork
biedt de mogelijkheid om een document, de hoofdstuk indeling etc. te beschrijven
in een Structured Chart. In de structured chart wordt verwezen naar het model
dat het in ontwikkeling zijnde systeem beschrijft. Een model is een verzameling
entity relation diagrams, data flow diagrams, state transition diagrams, data
dictionary entries etc. DPI biedt de mogelijkheid uit de structured chart en het
model een standaard document te produceren in Interleaf, Scribe of VAXDocument
formaat, maar bleek ook aanpasbaar en uitbreidbaar. En dat hebben we gedaan. DPI
produceert nu ook \LaTeX\ source files. De diagrammen worden geproduceerd in
PostScript en ingevoegd in het document waarvan de structuur, de documentstyle
etc. in \LaTeX\ gedefinieerd is volgens de geldende normen.

Een ander voorbeeld is een tool, DataModel, waarmee het mogelijk is databases te
ontwerpen. Met behulp van invulschermen en een databestand wordt de opbouw van
de database vastgelegd of gewijzigd. De tool genereert twee dingen, de Oracle
statements waarmee de database daadwerkelijk gecre\"eerd kan worden en een
\LaTeX\ source file die voldoet aan de standaard voor een zogenaamd Database
Design Document met de complete benodigde beschrijving van diezelfde database.

\section{Toekomst}

Zoals uit het voorgaande moge blijken, zoekt het CAWCS het voornamelijk in de
toepassing van \LaTeX\ en in mindere mate \TeX. Ook in de toekomst zal de nadruk
liggen op het uitbreiden van de toepassingsmogelijkheden.  Behoud van wat er is
vormt de belangrijkste overweging die speelt, bijvoorbeeld bij de upgrade naar
een volgende versie van \TeX. Er zal een grondige evaluatie aan voorafgaan.

Er zijn ook plannen en idee\"en voor verdere ontwikkeling van tools en/of
uitbreiding van bestaande. LSEDIT biedt sinds kort de mogelijkheid gebruik te
maken van een Program Design Language en daaruit documenten te genereren volgens
de Dod-STD-2167A, een documentatie norm waaraan ook het CAWCS moet voldoen.
ASCII text en Runoff source zijn de mogelijke vormen van een dergelijk document,
maar dat is uitbreidbaar en wat ons betreft is \LaTeX\ daarvoor een uitstekende
candidaat.

Ook volledige ondersteuning, in de vorm van \LaTeX\ templates in LSEDIT, voor de
complete DoD-STD-2167A documentatie set behoort tot de toekomstplannen.

Een ander idee is een Ada source code formatter, waaruit niet alleen
geformateerde (compileerbare) Ada source files komen maar ook \LaTeX\ source
files, waarmee dan een `pretty printer' gerealiseerd is.

Verder zullen de toepassingsmogelijkheden van SGML en aanverwanten bezien
worden.

\section{Conclusie}

Zoals gezegd: Wij zijn niet ge\"{\i}nteresseerd in \TeX\ en \LaTeX,

\vspace{2cm}

\begin{center}
{\LARGE Maar wel in wat het voor ons kan doen!}
\end{center}

\end{document}
