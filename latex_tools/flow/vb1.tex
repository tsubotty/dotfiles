\documentstyle[CMacs,nassi]{qadoc}
\docno{NTG-voorbeeld-001}
\author{Leo en Marion van Geest}
\date{28 februari 1991}
\title{Voorbeeld van verschillende}
\subtitle{CAWCS styles en opties}
\class{Niet geclassificeerd}
\begin{document}

\section{De document style QADoc}

Dit document is tot stand gekomen door de volgende commando's :

\begin{verbatim}
    \documentstyle[CMacs,nassi,nl]{QADoc}
    \docno{NTG-voorbeeld-001}
    \author{Leo en Marion van Geest}
    \date{28 februari 1991}
    \title{Voorbeeld van verschillend}
    \subtitle{CAWCS styles en opties}
    \class{Niet geclassificeerd}
\end{verbatim}

De style QADoc verzorgt zowel de documentstyle als de pagestyle, standaard in
11~pts. De parameters voor de heading worden in de pre-amble gedefinieerd.

Behalve de layout valt bij deze style ook op dat elke sectie op een nieuwe
bladzijde begint, subsecties daarentegen niet. Alle standaards als table of
contents, list of figures, glossary, index etc. kunnen gebruikt worden. Elk van
deze onderdelen komt in een aparte sectie.

\section{De optie CMacs}
De optie CMacs bevat onder andere de definities voor `Bits' die gebruikt wordt
om bitpatronen en data structuren te vertonen.

Een simpel voorbeeld is:

\beginbits{VDU SWITCHED}
  \word{1}{source=22 (MH1 active)}{H1}
  \bytes{2}{type=1}{subtype=3}{H2}
  \word{3}{VDU number}{ID}
\endbits

wat bereikt wordt met de commando's:

\begin{verbatim}
\beginbits{VDU SWITCHED}
  \word{1}{source=22 (MH1 active)}{H1}
  \bytes{2}{type=1}{subtype=3}{H2}
  \word{3}{VDU number}{ID}
\endbits
\end{verbatim}

De parameter van \verb"\beginbits" is de naam van de structuur. De parameters
van \verb"\word" en \verb"\byte" geven de mogelijkheid om text voor, in en na de
structuur-regel te specificeren.


\section{De optie Nassi}
De optie Nassi maakt het mogelijk Nassi-Schneidermann diagrammen te maken.

Het volgende diagram

\scriptsize
\STRUCT{Structure name}{Structure purpose}{%
  \ACTION{initial statement}%
  \PROC{proc name}{procedure purpose}%
  \IF{condition to test}%
  \THEN{%
    \ACTION{true action 1}%
    \ACTION{true action 2}%
  }%
  \ELSE{%
    \ACTION{false action}%
  }%
  \ENDIF%
  \REPEAT{%
    \ACTION{statement to repeat}%
  }%
  \UNTIL{end condition}%
  \WHILE{start condition}{%
    \ACTION{statement to do}%
  }%
  \ENDWHILE%
  \CASE{case item}{%
    \WHEN{condition 1}{%
      \ACTION{statement to do}%
    }%
    \WHEN{condition 2}{%
      \ACTION{statement 1 to do}%
      \ACTION{statement 2 to do}%
    }%
    \WHEN{condition 3}{%
      \ACTION{statement to do}%
    }%
  }%
  \ENDCASE%
}%
\normalsize

\vspace{0.5cm}

wordt bereikt door de commando's:

\begin{verbatim}
\scriptsize
\STRUCT{Structure name}{Structure purpose}{%
  \ACTION{initial statement}%
  \PROC{proc name}{procedure purpose}%
  \IF{condition to test}%
  \THEN{%
    \ACTION{true action 1}%
    \ACTION{true action 2}%
  }%
  \ELSE{%
    \ACTION{false action}%
  }%
  \ENDIF%
  \REPEAT{%
    \ACTION{statement to repeat}%
  }%
  \UNTIL{end condition}%
  \WHILE{start condition}{%
    \ACTION{statement to do}%
  }%
  \ENDWHILE%
  \CASE{case item}{%
    \WHEN{condition 1}{%
      \ACTION{statement to do}%
    }%
    \WHEN{condition 2}{%
      \ACTION{statement 1 to do}%
      \ACTION{statement 2 to do}%
    }%
    \WHEN{condition 3}{%
      \ACTION{statement to do}%
    }%
  }%
  \ENDCASE%
}%
\normalsize
\end{verbatim}

Door toevoeging van enige `dimensie' commando's kan in breedte, en daarmee in
hoogte gevarieerd worden. Met het commando \verb"\nassiwidth=\textwidth" wordt
het volgende resultaat bereikt:

\nassiwidth=\textwidth
\scriptsize
\STRUCT{Structure name}{Structure purpose}{%
  \ACTION{initial statement}%
  \PROC{proc name}{procedure purpose}%
  \IF{condition to test}%
  \THEN{%
    \ACTION{true action 1}%
    \ACTION{true action 2}%
  }%
  \ELSE{%
    \ACTION{false action}%
  }%
  \ENDIF%
  \REPEAT{%
    \ACTION{statement to repeat}%
  }%
  \UNTIL{end condition}%
  \WHILE{start condition}{%
    \ACTION{statement to do}%
  }%
  \ENDWHILE%
  \CASE{case item}{%
    \WHEN{condition 1}{%
      \ACTION{statement to do}%
    }%
    \WHEN{condition 2}{%
      \ACTION{statement 1 to do}%
      \ACTION{statement 2 to do}%
    }%
    \WHEN{condition 3}{%
      \ACTION{statement to do}%
    }%
  }%
  \ENDCASE%
}%
\normalsize

\end{document}

