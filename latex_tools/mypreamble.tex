% パッケージの設定 http://be.nucl.ap.titech.ac.jp/~sako/TeXmacro.pdf
\usepackage{color}                                % 色を付ける graphicxの前に配置しないとまずい
\usepackage[dvipdfmx]{graphicx}
% \usepackage{lscape}                               % 横にできる http://www.lightstone.co.jp/latex/kb0036.htm
\usepackage{subfigure}                            % 図にsubcaptionを付けれる http://did2.blog64.fc2.com/blog-entry-239.html
\usepackage{multicol}                             % 二段組など http://www.biwako.shiga-u.ac.jp/sensei/kumazawa/tex/multicol.html
\usepackage{ascmac}                               % screen, itembox, shadebox などが使えるようになる
\usepackage{here}                                 % 強制的に現在位置に図表を配置 http://toriaezuyattemiru.cocolog-nifty.com/blog/2010/06/latex-heresty-6.html
\usepackage{listings, jlisting}                   % ソースコードを綺麗に表示 jlistingで日本語対応 http://d.hatena.ne.jp/mallowlabs/20061226/1167137637
\usepackage{eclclass}                             % ツリーを作成出来る http://www.rsch.tuis.ac.jp/~mizutani/online/latex/style.html
\usepackage{Flow}                                 % フローチャート file:///usr/local/texlive/texmf-local/tex/latex/local/flow/flowchrt.pdf
\usepackage{layout}                               % 本文中に\layoutでレイアウトを可視
\usepackage{amssymb}                              % http://www.biwako.shiga-u.ac.jp/sensei/kumazawa/tex/amssymb.html
\usepackage{amsmath}                              % http://keizai.xrea.jp/latex/tutorial/ams.html
\usepackage[english]{babel}                       % http://d.hatena.ne.jp/xr0038/20080918/1221720101
\usepackage{txfonts}                              % TX fontsを使う 他のパッケージの後に読み込むように

%\makeatletter
%\def\tbcaption{\def\@captype{table}\caption}
%\def\figcaption{\def\@captype{figure}\caption}
%\makeatother

% Listingsの設定
\lstset{                                                   % http://www.biwako.shiga-u.ac.jp/sensei/kumazawa/tex/listings.html
  language=C,                                              % 言語設定
  numbers=left,
  stepnumber=1,
  numberstyle=\tiny,
	%numbersep=
  breaklines=true,                                         % 行が長い時に改行するか否か
	%breakindent=,                                           % 改行時インデント量
  basicstyle=\ttfamily\scriptsize,
  %commentstyle={\itshape \color[cmyk]{1,0.4,1,0}},
  classoffset=1,
  %keywordstyle={\bfseries \color[cmyk]{0,1,0,0}},
  %stringstyle={\ttfamily \color[rgb]{0,0,1}},
  frame=tRBl,
  framesep=5pt,
  showstringspaces=false,
  tabsize=2
}
